\documentclass[journal]{aiaa-pretty} % submit, draft, journal options can be used

\usepackage[belowskip=0pt, aboveskip=0pt]{subcaption}
\usepackage{float}
\usepackage{caption}

%\setlength{\belowcaptionskip}{-10pt}

\usepackage{amsmath}
\newenvironment{rcases}
  {\left.\begin{aligned}}
  {\end{aligned}\right\rbrace}
\newenvironment{lcases}
  {\left\lbrace\begin{aligned}}
  {\end{aligned}\right.}

\usepackage{amssymb}
\usepackage{amsfonts}
\usepackage{multirow}
\usepackage{booktabs}
\usepackage{doi}
\usepackage{xcolor}
\usepackage{graphicx,dblfloatfix} 
\usepackage{algorithm}
\usepackage{algpseudocode}
%\usepackage[superscript]{cite}
\usepackage[numbers,compress,sort]{natbib}
%
\usepackage{soul,xcolor}
\setstcolor{blue}

% User defined commands
\newcommand{\ds}{\displaystyle}
\newcommand{\mb}{\mathbf}
\newcommand{\mr}{\mathrm}
\newcommand{\mbs}{\boldsymbol}
\newcommand{\f}{\frac}
\newcommand{\p}{\partial}
\newcommand{\vect}[1]{\vec{\mathbf{#1}}}
\newcommand{\ignore}[1]{}

\renewcommand{\pd}[2]{\displaystyle{\dfrac{\partial #1}{\partial #2}}}
\newcommand{\td}[2]{\dfrac{d #1}{d #2}}
\newcommand{\pdt}[2]{\dfrac{\partial^2 #1}{\partial #2^2}}
\newcommand{\tdt}[2]{\dfrac{d^2 #1}{d #2^2}}
\newcommand{\pdtno}[2]{\dfrac{\partial^2 #1}{\partial #2}}
\newcommand{\pdd}[3]{\dfrac{\partial^2 #1}{\partial #2 \partial #3}}
\newcommand{\pff}[3]{\dfrac{d^2 #1}{d #2 d #3}}

% Redefined commands
\renewcommand\floatpagefraction{.95}
\usepackage{cleveref}
\crefname{subsection}{subsection}{subsections}

%\newcommand{\com}[1]{\textcolor{red}{#1}}
%\newcommand{\gjk}[1]{{\leavevmode\color{red}{#1}}}
\newcommand{\kb}[1]{{\leavevmode\color{blue}{#1}}}
\newcommand{\off}[1]{}

\graphicspath{{figures/}}

\title{Getting Started with Coarray Fortran}

\author{Komahan Boopathy \thanks{Ph.D. Candidate, School of Aerospace
    Engineering, 270 Ferst Drive, \url{komahan@gatech.edu}, AIAA
    Student Member.} \\
  {\normalsize\itshape Georgia Institute of Technology,
    Atlanta, GA, 30332-0150, USA.}\\
} 

\begin{document}

\abstract{abstract.}

\maketitle

\section{Introduction and Motivation}

Scientific programmers do not have to worry about parallelizing the
work.

A Fortran program containing coarrays is interpreted as if it were
replicated a fixed number of times and all copies were executed
asynchronously. Each copy has its own set of data objects and is
called an image. The array syntax of Fortran is extended with
additional trailing subscripts in square brackets to give a clear and
straightforward representation of access to data on other images.
References without square brackets are to local data objects, so code
that can run independently is uncluttered. Any occurrence of square
brackets in dicates communication between images, which might be slow.

-- Parallel programming importance
-- Languges and libraries that support parallel programming
   - shared memory/distributed memory
   - PGAS languages

Fortran programming language had been in existence since 1960s. The
language that boasts a standard's committee whose target audience is
scientific programmers.

Corray fortran had been an extension to the language and was first
developed by Numrich and Reid. It has been invoked as a part of the
language standard since 2008.

what are images?
what underlies this?
compiler support?

\section{Installation}

\section{Problem}

\section{Implementation}

\section{Results}

\bibliographystyle{myabbrvnat}
\bibliography{citations}

\end{document}
